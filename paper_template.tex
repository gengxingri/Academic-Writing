\documentclass[twoside,10pt]{article}

%\usepackage{refcheck}

\usepackage[pagewise]{lineno}%\linenumbers

\usepackage[english]{babel}
\usepackage{amsmath}
\usepackage{amsthm}
\usepackage{amsfonts}
\usepackage{amssymb}
\usepackage{graphicx}
\usepackage{colortbl,dcolumn}
\usepackage{paralist}  % generate "compactenum"
\usepackage{latexsym}
\usepackage{amsfonts}
\usepackage{amssymb}
\usepackage{cite}
\usepackage{appendix}
\usepackage{subcaption}
\usepackage{tikz}

     %\textheight=130truemm
     \textheight=220truemm
%
     %\textwidth=158truemm
     \textwidth=160truemm

\hoffset=0truemm
\voffset=0truemm

\topmargin=0truemm
% \topmargin=-25truemm
%\topmargin=-10truemm
     %\headheight=12pt
     %\headsep=15pt
     \oddsidemargin=0truemm
     \evensidemargin=0truemm

     %\theoremstyle{plain}
     \newtheorem{lemma}{\bf Lemma}[section]
     \newtheorem{theorem}{\bf Theorem}[section]
     \newtheorem{proposition}{\bf Proposition}[section]
     \newtheorem{corollary}{\bf Corollary}[section]
     \newtheorem{definition}{\bf Definition}[section]
     \newtheorem{remark}{\bf Remark}[section]
     %\theoremstyle{remark}
     \newtheorem{example}{\bf Example}
     \numberwithin{equation}{section}
     \newtheorem{claim}{\bf Claim}[section]




%\newcommand{\R}{\mathbb{R}}
%\newcommand{\om}{\Omega_1}
%\newcommand{\on}{\Omega_2}
%\newcommand{\de}{\delta}
%\newcommand{\be}{\beta}
%\newcommand{\ga}{\gamma/\alpha}
%\newcommand{\sd}{\frac{\sigma}{\delta}}
%\newcommand{\ssm}{\sqrt{\sigma\mu}}
%\newcommand{\g}{\gamma}
%\newcommand{\al}{\alpha}
%\newcommand{\po}{\partial\Omega_1}
%\newcommand{\n}{\mathbf{n}}
%\newcommand{\s}{\mathbf{s}}
%\newcommand{\tw}{\int_0^T\int_{\Omega_1}}
%\newcommand{\tv}{\int_0^T\int_{\Omega_2}}
%\newcommand{\la}{\lambda}
%\newcommand{\kk}{{\color{blue}k}}


\newcommand\eqnref[1]{(\ref{#1}}
\newcommand{\thmref}[1]{Theorem~\ref{#1}}
\newcommand{\secref}[1]{\S\ref{#1}}
\newcommand{\lemref}[1]{Lemma~\ref{#1}}
%\newcommand\bes{\begin{eqnarray}}
%\newcommand\ees{\end{eqnarray}}
%\newcommand\bess{\begin{eqnarray*}}
%\newcommand\eess{\end{eqnarray*}}
%\newcommand\cL{{\mathbf L}}
\newcommand\ccL{{\mathcal L}}
\newcommand{\bsu}{{\boldsymbol{u}}}
\newcommand{\bsv}{\boldsymbol{v}}
\newcommand{\bsw}{\boldsymbol{w}}
\newcommand{\bsf}{\boldsymbol{f}}
\newcommand{\bsm}{\boldsymbol{m}}
\newcommand{\bsg}{\boldsymbol{g}}
\newcommand{\bszero}{\boldsymbol{0}}
\newcommand{\bsone}{\boldsymbol{1}}
\newcommand{\bsvarphi}{\boldsymbol{\varphi}}
\newcommand{\bspsi}{\boldsymbol{\psi}}
\newcommand{\dx}{{\mathrm d}x}
\newcommand{\ds}{{\mathrm d}s}

\newcommand{\dt}{{\mathrm d}t}
\newcommand{\bsc}{\boldsymbol{c}}
%\newcommand{\lspace}{-6pt}

\begin{document}
\title{{\LARGE PAPER TITLE}
 \footnotetext{
 E-mail addresses: ...; ...\\ $^*$Corresponding author}}


\author{{The first author $^{a,b}$ and the second author $^{b,*}$}\\[2mm]
\small\it $^a$ Institution, City, Country \\
\small\it $^b$ Institution, City, Country}


\date{}


\maketitle


\begin{abstract}
This is our abstract

...

...

...
\end{abstract}

\noindent{\bf Keywords.} several keywords...\\

\noindent{\bf AMS subject classifications.} ...

%\tableofcontents

\section{Introduction}
This is our introduction part 

...

...

...


\begin{equation}\label{Intro:1}
\left\{
	\begin{array}{rcll}
		\partial_{t} u_i-\Delta u_i &=& \lambda g_i(x)f(u_i),\;\; & (x,t)\in \Omega_i\times (0,\infty),\\
		\nu\cdot \nabla u_i&=&\gamma (u_2-u_1),& (x,t)\in \Sigma\times (0,\infty),\\
\nu \cdot \nabla u_2 &= &0,\;\; & (x,t)\in \Gamma\times (0,\infty),\\
		u_i(x,0)&=&u_{i0}(x),& x\in \Omega_i.
	\end{array}
	\right.
\end{equation}


\begin{figure}[h]
%\centerline{\scalebox{0.5}{\includegraphics{f1.png}}}
\caption{Domains $\Omega_1$ and $\Omega_2$, interface $\Sigma$,  outer boundary $\Gamma$, and unit outer normal vector $\nu$.}
\end{figure}










This paper is organized as follows.

...

...

...


\section{Section two}

This is our section two

...

...

...



\subsection{Subsection one}\label{subsec:} 




\begin{definition}\label{def:1}
We call $\bsu$  a weak solution of \eqref{Intro:1} if $\bsu\in W_{2}^{1,0}$ and
	\begin{equation}
		-\int _0^T (\bsu,\boldsymbol{\xi}_t)_{L^2}\; \dt  + \int_0^T B_\gamma (\bsu,\boldsymbol{\xi}) \; \dt - (\bsu_0,\boldsymbol{\xi}(\cdot,0))_{L^2}= \int _0 ^T (\bsf,\boldsymbol{\xi})_{L^2}\;\dt
	\end{equation}
	for every  $\boldsymbol{\xi}\in W_2^{1,1}$ satisfying $\boldsymbol{\xi}(\cdot,T)\equiv 0$ in $\Omega$.
\end{definition}

\begin{remark}{\rm According to \cite[Theorem\,2]{Chen2001},
	our Definition\,\ref{def:1} is in accordance with the definitions of weak solutions in  \cite[Definition\,2.1 and (16)]{Chen2001}.}
\end{remark}

...

...

...

\begin{theorem}\label{existence:1}
	If $\bsc\in L^\infty$ and $\bsf\in L^2((0,T);L^2)$,
	then \eqref{Intro:1} has a unique weak solution $\bsu$ for every $\bsu_0\in L^2$. Moreover,
	 $\bsu\in V^{1,0}_2$ and $\bsu(\cdot,t)\to \bsu_0$ in $L^2$ as $t\to 0+$.
\end{theorem}



\begin{proof}
	
	\begin{eqnarray}
		B_\gamma (\bsu,\bsv)&\le& C_1 ||\bsu||_{W_2^1}||\bsv||_{W_2^1}\;\;\forall\, \bsu,\bsv\in W_2^1,\label{1127}\\
		B_\gamma(\bsu,\bsu)&\ge& C_2 ||\bsu||_{W_2^1}^2\;\;\forall\, \bsu\in W_2^1.\label{1128}
	\end{eqnarray}

...

...

...
	
	

\end{proof}

\begin{corollary}
	Suppose $\bsc\in L^\infty$ and $\bsf\in L^2$. If $\bsu$ is a weak solution of \eqref{Intro:1}, then there exists a constant $C=C(\bsc)$ such that
	\begin{equation}
		||\bsu||_{W_2^1}\le C \left(||\bsf||_{L^2} + ||\bsu||_{L^2}\right).
	\end{equation}
\end{corollary}

\subsection{Subsection two}\label{subsec:2}














\begin{lemma}\label{lem:steklov}
	If $\bsu\in V_{2,T}^{1,0}$, then $\bsu^h\in W^{1,1}_{2,T-\delta}$ for all $0<h\le \delta$ and
	\begin{equation}
		\bsu^h\to \bsu \;\;\; \mbox{in}\;\;\; V_{2,T-\delta}^{1,0}\;\;\mbox{as}\;\; h\to 0+.
	\end{equation}
\end{lemma}
\begin{proof}
This lemma follows from applying \cite[p.\,85, Lemma\,4.7]{LSU1968} to each $u_i$ for $i=1,2$.
\end{proof}




	
	
	





\begin{theorem}\label{weaksol:1}
	Suppose $\bsc\in C^{\alpha}$ for some $\alpha\in(0,1)$ and $\bsf\in L^\infty((0,T); L^\infty)$. The following conclusions hold for the unique weak solution of \eqref{Intro:1}.\\
\noindent\emph{(i)} If $\bsu_0\in L^\infty$, then $\bsu\in L^\infty((0,T);L^\infty)$ and
\begin{equation}\label{eq:bound}
			||\bsu||_{L^\infty((0,T);L^\infty)}\le C \left( ||\bsu_0||_{L^\infty} + ||\bsf||_{L^\infty((0,T);L^\infty)}\right).
		\end{equation}
\noindent\emph{(ii)} If $\bsu_0\in L^\infty$, then
	   \begin{equation}\label{eq:holder}
	   	u_i\in C^{\alpha,\frac{\alpha}{2}}(\bar{\Omega}_i\times [\epsilon,T])\;\;\mbox{for all} \; \; \epsilon\in (0,T), \;i=1,2.
	   \end{equation}
\noindent\emph{(iii)} If $\bsu_0\in C^0$, then $\bsu\in C^0([0,T];C^0)$ and
	   \begin{equation}
	   	\bsu(\cdot,t)\to \bsu_0\;\;\mbox{in}\;\; C^0\;\;\mbox{as}\;\; t\to 0+.
	   \end{equation}
\end{theorem}

...

...

...



\subsection{Subsection three}\label{subsec:3}



...

...

...


\section{Section three}

The first eigenvalue problem reads
\begin{equation}\label{EV:1}
	\left\{
	\begin{array}{rcll}
		-\Delta \psi_i + c_i(x)\psi_i &=& \mu \psi_i, \;\;& x\in\Omega_i,\\
		\nu\cdot\nabla\psi_i &=&\gamma (\psi_2-\psi_1),&x\in\Sigma,\\
		\nu\cdot \nabla \psi_2 &=&0,\;\; &x\in \Gamma,
	\end{array}
	\right.
\end{equation}


\begin{figure}
%\centerline{\scalebox{0.5}{\includegraphics{test.jpg}}}
\caption{The construction of $\tilde\bsvarphi$ from $\bsvarphi_{b}$, where the dropped height $h=\bsvarphi_{b}'(b)/\gamma$.}
\end{figure}

...

...

...



\section{Discussion}


...

...

...


\bigskip
\section{ Acknowledgment*} ...
\newpage

%-------------Appendix--------------%
\appendix
\begin{center}
	\bfseries Appendix
\end{center}
\setcounter{theorem}{0}
\setcounter{equation}{0}
\renewcommand{\thetheorem}{A\arabic{theorem}}
\renewcommand{\theequation}{A.\arabic{equation}}


\begin{equation}\label{Apx:1}
	\left\{
	\begin{array}{rcll}
		\partial_t w-\Delta w + c(x,t)w&=& f(x,t),\;\; &(x,t)\in D_T, \\
		\nu\cdot \nabla w &=&\psi(x,t), &(x,t)\in (\partial D)_T,\\
w(x,0)&=&w_0(x),&x\in D.
	\end{array}
	\right.
\end{equation}

\begin{theorem}\emph{(\cite[Theorem\,17]{Sol64})} \label{thm:apx1}
	Suppose that $c\in C^{\alpha,\frac{\alpha}{2}}(\overline{D_T})$ and that $f\in L^p(D_T)$, $w_0\in W_p^{2-\frac{2}{p}}(D)$, and $\psi \in W_p^{1-\frac{1}{p},\frac{1}{2}-\frac{1}{2p}}((\partial D)_T)$ for some $p\in (1,\infty)$.
 When $p>3$, we assume that, in the sense of trace,
 $$\nu \cdot \nabla w_0=\psi(\cdot,0) \mbox{~on~} \partial D.$$
Then  \eqref{Apx:1} admits a unique strong solution $w\in W_p^{2,1}(D_T)$, which satisfies
	\begin{equation}\label{B.2}
		||w||_{W^{2,1}_p(D_T)} \le  C \left( ||f||_{L^p(D_T)} + ||w_0||_{W^{2-\frac{2}{p}}_p(D)} + ||\psi||_{W^{1-\frac{1}{p},\frac{1}{2}-\frac{1}{2p}}_p((\partial D)_T)}\right)
	\end{equation}
provided $p\neq 2,3$.
\end{theorem}


\begin{theorem}\emph{(\cite[Corollary\,5.1.22]{Lunardi1995})}\label{thm:apx2}
	Suppose that $c,f\in C^{\alpha,\frac{\alpha}{2}}(\overline{D_T})$, $w_0\in C^{2+\alpha}(\overline{D})$, and
	$\psi\in C^{1+\alpha,\frac{1}{2}+\frac{\alpha}{2}}(\overline{(\partial D)_T})$. Assume
$$\nu\cdot \nabla w_0=\psi(x,0)\;\;\mbox{on}\;\; \partial D.$$
 Then \eqref{Apx:1} has a classical solution $w\in C ^{2+\alpha,1+\frac{\alpha}{2}}(\overline{D_T})$, which satisfies
	\begin{equation}\label{B.3}
		||w||_{C^{2+\alpha,1+\frac{\alpha}{2}}(\overline{D_T})} \le C
		\left(
		||w_0||_{C^{2+\alpha}(\overline{D})}+\|f\|_{C^{\alpha,\frac{\alpha}{2}}(\overline{D_T})}
		+ ||\psi||_{C^{1+\alpha, \frac{1}{2}+ \frac{\alpha}{2}}(\overline{(\partial D)_T})}
		\right).
	\end{equation}
\end{theorem}




\begin{theorem}\emph{(\cite[Theorem\,2.6 and Corrollary\,2.10]{SW2009})}

	\begin{equation}\label{eq:C2}
		||w||_{W^{2}_p(D)}\le C \left(
		||f||_{L^p(D)} + ||\psi||_{W^{1-\frac{1}{p}}_p(\partial D)}
		\right).
	\end{equation}
\end{theorem}
\begin{theorem}\emph{(\cite[Theorems\,3.1 and 3.2]{LN1968} and \cite[Proposition\,3.3]{Amann1976})}
\label{C:2}


\begin{equation}
	||w||_{C^{2+\alpha}(\overline{D})} \le C \left( ||f||_{C^{\alpha}(\overline{D})} +
	||\psi||_{C^{1+\alpha}(\partial D)}
	\right),
\end{equation}
\begin{equation}
	||w||_{W^1_p(D)} \le C_p
	\left(
	||f||_{L^p(D)} + ||\psi||_{L^p(\partial D)}
	\right)\quad \forall\, p\in (1,\infty).
\end{equation}
\end{theorem}




%-----------------------Reference------------------%
\begin{thebibliography}{100}\setlength{\itemsep}{-0.5mm}

%\begin{thebibliography}{100}

\bibitem{Adam}
\newblock R. A. Adams,
\newblock Sobolev Spaces,
\newblock Academic Press, New York, 1975.

\bibitem{Amann1976}
\newblock H. Amann,
\newblock Nonlinear Elliptic Equations with Nonlinear Boundary Conditions,
\newblock in: North-Holland Mathematics Studies, Elsevier, 1976: pp.\,43--63.
\bibitem{Amann2019}
\newblock H. Amann,
\newblock Linear and Quasilinear Parabolic Problems, Vol. II, Function Spaces,
\newblock Springer, 2019.


\bibitem{BCF80}
\newblock H. Br\'ezis, L. A. Caffarelli and A. Friedman, Reinforcement problems for elliptic equations and variational inequalities,
\newblock Ann. Mat. Pura Appl. 123 (1980) 219--246.


\bibitem{BR14}
\newblock R. B\"urger,
\newblock A survey of migration-selection models in population genetics,
\newblock Discrete Contin. Dyn. Syst. Ser. B 19 (2014) 883--959.



\bibitem{CC03}
R. S. Cantrell and C. Cosner, Spatial Ecology via Reaction-Diffusion Equations, Series in
Mathematical and Computational Biology, John Wiley and Sons, Chichester, UK, 2003.



\bibitem{Chen2001}
\newblock C.-K. Chen,
\newblock A barrier boundary value problem for parabolic and elliptic equation,
\newblock Communications in Partial Differential Equations 26 (2001) 1117--1132.

\bibitem{Chen2006}
\newblock C.-K. Chen,
\newblock A fixed interface boundary value problem for differential equations: a problem arising from population genetics,
\newblock Dynamics of PDE 3 (2006) 199--208.


\bibitem{PW1984}
\newblock M. H. Protter and H. F. Weiberger,
\newblock Maximum Principle in Differential Equations,
\newblock Springer-Verlag, Berlin, Second edition, 1984.


\bibitem{LSU1968}
\newblock O. A. Ladyzhenskaya, V. A. Solonnikov and N. N. Ural'tseva,
\newblock Linear and quasilinear equations of parabolic type,
\newblock Transl. Math. Monogr. 23 Amer. Math. Soc., Providence, RI, 1968.

\bibitem{LN1968}
\newblock O. A. Ladyzhenskaya and N. N. Ural'tseva,
\newblock Linear and quasilinear elliptic equations,
\newblock Academic Press, New York, 1968.










\end{thebibliography}




\end{document}
